\section{Introduction}

The goal of this project is to replicate the work done in \citeauthor{daly2015hand} \cite{daly2015hand}. As described in the paper, our goal is to produce a system that can reproduce graphs from hand-drawn sketches. \citeauthor{daly2015hand} does not have a public implementation, so we are reproducing the implementation ourselves from scratch. The \href{https://github.com/neerajgangwar/graph-recognition}{Github repository} contains iPython notebooks for each subtask that can be used to reproduce the results presented in this report.\\

The recognition process has three main steps - segmentation, classification, and domain interpretation. We have described these steps in their respective sections. Segmentation is discussed in section \ref{sec:segmentation}, classification in section \ref{sec:classification}, and the domain interpretation in section \ref{sec:interpretation}. \\

We consider two approaches to this problem. The method described by \citeauthor{daly2015hand}, where recognition happens as the user draws, is the online approach. As an extension to the original goal, we also attempted to implement an offline approach where the entire graph is processed at once. We could not find any preexisting work relating to this topic for the offline approach as a whole, although there are some preexisting works for individual steps that we describe in their sections.
