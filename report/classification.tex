\section{Classification}

After segmentation is completed, we run classification on each segment. This is to determine the probability that a given segment is one of four types - vertex, self-loop, edge, or arrow. These probabilities are used in domain interpretation step to reconstruct the original graph. 

\subsection{Algorithm}

In order to determine the segment's probability distribution, we find five parameters. These parameters were chosen by \citeauthor{daly2015hand} \cite{daly2015hand}, based on the alpha shape and convex hull of the segment. \\ 

To compute the alpha shape and convex hull, we use the Alpha Shape Toolbox library \cite{alphashapetoolbox}, as well as SciPy's ConvexHull implementation \cite{scipy}. These libraries provide tools to easily find the area, perimeter, and points of these shapes. \\

\subsubsection{Circumscribed circle}

The first parameter is used to distinguish vertices and loops from other segments. This parameter $x_1$ is computed as \\

\begin{equation}
	x_1 = \frac{\text{Area of Convex Hull}}{\text{Area of circumscribed circle of Convex Hull}}
\end{equation} \\

The convex hull is not necessarily cyclic, so we define the circumscribed circle as a circle with diameter equal to the distance between the two farthest points in the convex hull. \\

\subsubsection{Inscribed triangle}

The second parameter is used to distinguish arrows. This parameter $x_2$ is computed as \\

\begin{equation}
	x_2 = \frac{\text{Area of largest inner triangle with angles over 20\textdegree}}{\text{Area of Convex Hull}}
\end{equation} \\

We use the method by \citeauthor{largesttriangle} \cite{largesttriangle} to find the area of the largest inner triangle.

\subsubsection{Alpha shape ratio}

The third parameter is used to distinguish line segments. This parameter $x_3$ is computed as \\

\begin{equation}
	x_3 = \frac{\text{Perimeter of alpha shape with $\alpha$ = 25}}{500 \cdot \text{Area of alpha shape with $\alpha$ = 25}}
\end{equation} \\

\subsubsection{Perimeter}

The fourth parameter is used to distinguish vertices and arrows from other segments. This parameter $x_4$ is computed as \\

\begin{equation}
	x_4 = \frac{\text{Perimeter of Convex Hull}}{\text{500}}
\end{equation} \\

\subsubsection{Disjoint shapes}

The final parameter disqualifies any strokes that are not properly segmented. This parameter $x_5$ is computed as \\

\begin{equation}
	x_5 = \text{Number of disjoint regions in alpha shape over 50 pixels apart} - 1
\end{equation} \\

The weights applied to each parameter reflect traits held by classes of stroke. For example, vertices and self-loops are almost circular, so they assign high positive weight to the circumscribed circle parameter. For testing purposes, we use the original weights trained by \citeauthor{daly2015hand} \cite{daly2015hand}. Notably, the weight for parameter $x_5$ is not trained, and instead set to always be -1000, to minimize the probability distribution of any shape with multiple disjoint regions.

\subsection{Experiments}
Similarly to the segmentation algorithm, we tested classification by drawing strokes on the iPyCanvas. One example is shown in Figure \ref{fig:classification_example}.

\begin{figure}
	\centering
	\begin{subfigure}{0.9\textwidth}
		\centering
		\includegraphics[scale=0.5]{./img/classificationexample}
	\end{subfigure}
	\caption{Example segment. $p_{vertex} = 0.0000076\text{, }   p_{arrow} = 0.256\text{, } p_{edge} = 0.996\text{, }  p_{loop} = 0.000086$}
	\label{fig:classification_example}
\end{figure}

%\subsection{Future Work}
%
%Our future work is to connect the probability distributions computed in this step to the domain interpretation step. Certain classes tend to have similar probabilities, for example, vertices and self-loops are difficult to differentiate. These are typically differentiated in domain interpretation rather than classification. \\
%
%In addition, we may need to further test these implementations on more data, or train more on our own data so that our weights better reflect our specific implementations.

