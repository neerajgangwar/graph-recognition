\documentclass[10pt, twocolumn,letterpaper]{article}
%\usepackage[margin=1in]{geometry}
\usepackage{cvpr}
%\usepackage{graphicx}
\usepackage[font=small, labelfont=bf]{caption}
\usepackage{subcaption}
%\usepackage{hyperref}
%\usepackage{xcolor}
%\usepackage{nameref}
%\usepackage{multirow}
%\usepackage{amsmath}
%\usepackage{mathtools}
\usepackage{times}
\usepackage{epsfig}
\usepackage{graphicx}
\usepackage{amsmath}
\usepackage{amssymb}
\usepackage{booktabs}
\usepackage[breaklinks=true,bookmarks=false]{hyperref}
\usepackage[sorting=none, maxcitenames=2]{biblatex}
\addbibresource{bibliography.bib}
\setlength{\parindent}{0pt}

\cvprfinalcopy

% Title Page
\title{Hand-Drawn Graph Recognition\\
	ECE-549 / CS-543\\
	Project Report}
\author{Debopam Sanyal (dsanyal2) \and Neeraj Gangwar (gangwar2) \and Bryan Huang (bryanh2)}
% \date{}


\begin{document}
	\maketitle
	
	\section{Introduction}

The goal of this project is to replicate the work done in \citeauthor{daly2015hand} \cite{daly2015hand}. As described in the paper, our goal is to produce a system that can reproduce graphs from hand-drawn sketches. \citeauthor{daly2015hand} does not have a public implementation, so we are reproducing the implementation ourselves. \\

The recognition process has three main steps - segmentation, classification, and domain interpretation. We have described these steps in their respective sections. Segmentation is discussed in section \ref{sec:segmentation}, classification in section \ref{sec:classification}, and the domain interpretation in section \ref{sec:interpretation}. \\

We consider two approaches to this problem. The method described by \citeauthor{daly2015hand}, where recognition happens as the user draws, is the online approach. As an extension to the original goal, we also attempted to implement an offline approach where the entire graph is processed at once. We could not find any preexisting work relating to this topic for the offline approach as a whole, although there are some preexisting works for individual steps that we describe in their sections.

	
	\section{Segmentation}
The first step towards graph recognition is segmentation. This step identifies the segmentation points in the input graph. These points are fed into the recognition task to identify a component. Our implementation is based on the idea presented by \citeauthor{daly2015hand} \cite{daly2015hand} and assumes that the entire component is drawn in one stroke. However, drawing multiple components in the same stroke is acceptable for our implementation.\\

The segmentation points identify the transition from one component to the next. These are referred to as true segmentation points.  However, there are certain challenges in the segmentation task. These challenges arise from the fact that there might be additional points in a component with similar geometric properties as true segmentation points. This is shown in Figure \ref{fig:seg_false_neg}. The challenges can be divided into two buckets: false positives and false negatives. The points that are incorrectly identified as segmentation points contribute to false positives. These cases are less severe as they can be handled in the classification step. The points that are true segmentation points but are missed by the segmentation algorithm are false negatives. These cases are more severe as we cannot recover from these errors. The goal of the segmentation algorithm would be to minimize false positives while ensuring that there are no false negatives.\\

\begin{figure}
	\centering
	\begin{subfigure}{0.3\textwidth}
		\centering
		\includegraphics[scale=0.2]{./img/seg_orig.jpg}
	\end{subfigure}
	\begin{subfigure}{0.3\textwidth}
		\centering
		\includegraphics[scale=0.2]{./img/seg_true_seg.jpg}
	\end{subfigure}
	\begin{subfigure}{0.3\textwidth}j
		\centering
		\includegraphics[scale=0.2]{./img/seg_false_seg.jpg}
	\end{subfigure}
	\caption{\textit{left} a line and arrow head drawn in one stroke, \textit{middle} true segmentation point, \textit{right} additional points with similar properties as true segmentation points}
	\label{fig:seg_false_neg}
\end{figure}

\subsection{Algorithm}
The algorithm to identify segmentation points is based on the observation that there are abrupt changes in curvature at these points. This is based on the properties of transition that can occur in the three types of graph components. This is shown in figure \ref{fig:seg_curvature}.\\

Before the curvature calculation, some pre-processing of input data is required. We are using iPyCanvas to capture the coordinates of the drawn components. The number of coordinates captured depends on the drawing speed. If the drawing speed is higher in certain region, a number of points captured will be higher in that region. This count is normalized by considering the distance between two points. If the Euclidean distance between two consecutive points, $P_i$ and $P_{i+1}$, is less than a threshold, $d_t$, we drop $P_{i+1}$. Here, $d_t$ is a hyperparameter which needs to be trained based on the input data. For our usecase, $d_t = 1$ works well.\\

The curvature of a point $P_i$ is computed by calculating the angle between vectors from by joining $P_{i-s}$ and $P_i$ and $P_i$ and $P_{i+s}$. Here, $s$ is the second hyperparameters. For our usecase, we have kept $s=3$ based on experiments.

\begin{equation}
	C_s(P_i) = \arccos \left(  \frac{\overrightarrow{P_{i-s} P_i} \cdot \overrightarrow{P_i P_{i+s}}}{ \Vert \overrightarrow{P_{i-s} P_i}  \Vert \; \Vert \overrightarrow{P_i P_{i+s}} \Vert  } \right) 
\end{equation} 

Once the curvature is computed for all captured points, we compute the abnormality. This metric is used in identifying the abrupt and significant changes in the curvature. The abnormality of a point $P_i$ is defined by how the curvature at that point differs from the average curvature of the surrounding points.
\begin{equation}
	A_{s,w}(P_i) = C_s(P_i) - \frac{\sum_{j=a}^{i-1} C_s(P_j) + \sum_{j=i+1}^{b} C_s(P_j)}{b - a}
\end{equation}
where $a = max(1, i-w)$ and $b = min(n, i+w)$.\\

$w$ is the third hyperparameter. For our usecase, we use $w=5$ based on the experiments. Next, we need to find ranges of points where $k A_{s, w}(P_i) > 1$ and select the point with maximum abnormality in each range as the segmentation point. The multiplication factor is the fourth hyperparameter.\\

\begin{figure}
	\centering
	\begin{subfigure}{0.9\textwidth}
		\centering
		\includegraphics[scale=0.5]{./img/seg_curvature_plot}
	\end{subfigure}
	\caption{Curvature plot for the components shown in figure \ref{fig:seg_false_neg}. The peaks represent the segmentation points.}
	\label{fig:seg_curvature}
\end{figure}

\subsection{Experiments}
To test the segmentation algorithm, we drew some components and shapes manually on iPyCanvas and validated the results. These are shown in figure \ref{fig:seg_results}.
\begin{figure}
	\centering
	\begin{subfigure}{0.8\textwidth}
		\centering
		\includegraphics[scale=0.6]{./img/seg_results_arrow.jpg}
	\end{subfigure}
	\begin{subfigure}{0.8\textwidth}
		\centering
		\includegraphics[scale=0.6]{./img/seg_results_rect.jpg}
	\end{subfigure}
	\begin{subfigure}{0.8\textwidth}
		\centering
		\includegraphics[scale=0.6]{./img/seg_results_pentagram.jpg}
	\end{subfigure}
	\caption{Detected segmentation points (highlighted in red).}
	\label{fig:seg_results}
\end{figure}

\subsection{Future Work}
The immediate next step is to integrate the segmentation module with the classification module. The current segmentation algorithm is tailored for the online usecase. The future work will be extend this algorithm for offline graphs.
	
	\section{Classification}

After segmentation is completed, we run classification on each segment. This is to determine the probability that a given segment is one of four types - vertex, self-loop, edge, or arrow. These probabilities are used in domain interpretation step to reconstruct the original graph. 

\subsection{Algorithm}

In order to determine the segment's probability distribution, we find five parameters. These parameters were chosen by \citeauthor{daly2015hand} \cite{daly2015hand}, based on the alpha shape and convex hull of the segment. \\ 

To compute the alpha shape and convex hull, we use the Alpha Shape Toolbox library \cite{alphashapetoolbox}, as well as SciPy's ConvexHull implementation \cite{scipy}. These libraries provide tools to easily find the area, perimeter, and points of these shapes. \\

\subsubsection{Circumscribed circle}

The first parameter is used to distinguish vertices and loops from other segments. This parameter $x_1$ is computed as \\

\begin{equation}
	x_1 = \frac{\text{Area of Convex Hull}}{\text{Area of circumscribed circle of Convex Hull}}
\end{equation} \\

The convex hull is not necessarily cyclic, so we define the circumscribed circle as a circle with diameter equal to the distance between the two farthest points in the convex hull. \\

\subsubsection{Inscribed triangle}

The second parameter is used to distinguish arrows. This parameter $x_2$ is computed as \\

\begin{equation}
	x_2 = \frac{\text{Area of largest inner triangle with angles over 20\textdegree}}{\text{Area of Convex Hull}}
\end{equation} \\

We use the method by \citeauthor{largesttriangle} \cite{largesttriangle} to find the area of the largest inner triangle.

\subsubsection{Alpha shape ratio}

The third parameter is used to distinguish line segments. This parameter $x_3$ is computed as \\

\begin{equation}
	x_3 = \frac{\text{Perimeter of alpha shape with $\alpha$ = 25}}{500 \cdot \text{Area of alpha shape with $\alpha$ = 25}}
\end{equation} \\

\subsubsection{Perimeter}

The fourth parameter is used to distinguish vertices and arrows from other segments. This parameter $x_4$ is computed as \\

\begin{equation}
	x_4 = \frac{\text{Perimeter of Convex Hull}}{\text{500}}
\end{equation} \\

\subsubsection{Disjoint shapes}

The final parameter disqualifies any strokes that are not properly segmented. This parameter $x_5$ is computed as \\

\begin{equation}
	x_5 = \text{Number of disjoint regions in alpha shape over 50 pixels apart} - 1
\end{equation} \\

The weights applied to each parameter reflect traits held by classes of stroke. For example, vertices and self-loops are almost circular, so they assign high positive weight to the circumscribed circle parameter. For testing purposes, we use the original weights trained by \citeauthor{daly2015hand} \cite{daly2015hand}. Notably, the weight for parameter $x_5$ is not trained, and instead set to always be -1000, to minimize the probability distribution of any shape with multiple disjoint regions.

\subsection{Experiments}
Similarly to the segmentation algorithm, we tested classification by drawing strokes on the iPyCanvas. One example is shown in Figure \ref{fig:classification_example}.

\begin{figure}
	\centering
	\begin{subfigure}{0.9\textwidth}
		\centering
		\includegraphics[scale=0.5]{./img/classificationexample}
	\end{subfigure}
	\caption{Example segment. $p_{vertex} = 0.0000076\text{, }   p_{arrow} = 0.256\text{, } p_{edge} = 0.996\text{, }  p_{loop} = 0.000086$}
	\label{fig:classification_example}
\end{figure}

%\subsection{Future Work}
%
%Our future work is to connect the probability distributions computed in this step to the domain interpretation step. Certain classes tend to have similar probabilities, for example, vertices and self-loops are difficult to differentiate. These are typically differentiated in domain interpretation rather than classification. \\
%
%In addition, we may need to further test these implementations on more data, or train more on our own data so that our weights better reflect our specific implementations.


	
	\section{Domain Interpretation}
\label{sec:interpretation}
After we compute the probabilities from the classification section, we need to produce a final interpretation of the graph in terms of edges, vertices, arrows, self-loops, connections between edges and vertices and connections between arrows and edges. To do this we cannot use the trivial method of partitioning the graph into segments (from our segmentation step) and simply solving for the partition that produces the largest sum of classification probabilities (from the classification step). This is mainly because there is significant interaction between components of a graph. The design that we follow rests on the idea that the connections between the components influence the selection of the best interpretation.\\

\subsection{Scoring Subgraphs}

We found it useful to rank candidate graphs by scores that are formed from a range of metrics like sum of classifier probabilities $g(p)$, total number of components $C(p)$, number of connections $D(p)$ and number of missing connections $M(p)$. Thus, the combined score is simply: $V_{A,B,C}(p) = g(p) + A.C(p) + B.D(p) + C.M(p)$ where $g(p) = \sum_{(a_i,b_i,c_i) \in p}f(a_i, b_i, t_i)$. The function $f$ is determined by the classification probabilities. The score however is 0 when any component in $p$ has a probability less than 0.2. To properly assess the connections, we had to establish a set of machine specifications for our components where we could find the distances, say, between an edge and vertex or an edge an arrow.\\

Write machine specification.The recognition step provided some unique challenges. Here, an exhaustive search for candidate graphs had to be done because in the classification step we returned probabilities of all possible components. Since there are four different components that we consider: arrow, edge, vertex, loop, any segment can be one of the four components. Hence, the total number of candidate graphs to be considered at any step in the online setting is $4^{n_{segments}}$. We obtain $n_{segments}$ from the segmentation step where segmentation points are calculated from strokes. For each segment once classification is done, we first create a Cartesian product of components to include all possible combinations. Then each combination is converted to conform with our machine specifications (please refer to interpretation.py in our code). Next each combination (also a subgraph) is added to the locked-in graph to form the subgraph at step $i$. The score function assesses each such subgraph and assigns a final score. This final score is calculated using the connections formed between components in the classified graph and using the sum of probabilities acquired from the classification step. We discard the sum of probabilities for any subgraph if any of its components had less than 0.2 probability. This essentially means we filter out the subgraphs with extremely unlikely components. *Equation of score* The candidate subgraphs are sorted in descending order of their final scores. The highest scoring subgraph is treated as the interpreted graph for that step. Subsequently, the interpreted graph becomes the locked-in graph for the next step of handling user strokes and the pipeline is run again to find the interpreted graph at the end of the step.\\

In addition to the classification probability, the scoring depends on the connections too. Since the number of components tend to be over-counted by this method, it has a negative coefficient. The number of connections made has a positive coefficient because ideally, we want our interpretation module to detect as many connections as possible. A slightly more complex metric is the number of missing connections. To calculate it, we must penalize the graphs for having edge points not connected to vertices, arrows not connected to edges and self-loop with different starting and end vertices. The penalty is ensured here again by a negative coefficient. Finding loops connected to different vertices was challenging as looking for the same loop in two different connections was not an easy task. We resolved this by doing an element-wise comparison of the loop points.\\

\subsection{Connections}

A vertex-edge connection is tested for every vertex and edge pair. We can use the center and radius of the vertex to find the distance from the endpoints of the edge as edges are given as a set of target points that represent a smooth line. Computing the Euclidean distance between the center and each of the two endpoints of the edges gives us an idea of whether the connection is plausible. Of course, we must subtract the radius of the circle itself and threshold the resulting distance to discard the large distance values. We maintain a score q for each edge end point that is initialized to infinity and updated if and only if a new vertex is found for the end point such that it's distance falls below the threshold and it is less than the previous q (negative values are clipped to 0).\\

The arrow-edge connection algorithm is more complex because it involves triangles (machine specification for arrows) and lines (edges) that must intersect one of the sides of the triangle. Again, connection is tested for every arrow and edge pair (from the endpoints of the edge). Here, we first extend the edge sequence by adding a target point before the end point such that smoothness is still maintained (use the slope for this). The distance between the new and the old point is $\gamma$ . Now, we test for the segment between each consecutive pair of the extended edge. If the segment intersects with a side of the triangle, then two conditions must be satisfied. The first is that the distance between the endpoint and the point of intersection must be under the threshold or the distance between the endpoint and the point opposite to the intersection side of the triangle must be under the threshold. The second condition is valid only for arrows that were initially two-sided. It states that the side involved in the intersection is the same as the one that was initially non-represented by the user's drawing.\\

The central portion of domain interpretation is calculating the vertex-edge and arrow-edge connections. The vertex-edge calculation was relatively simple since the center is provided by the classification module. The next steps involve comparing the distances between the edge endpoints and the center of the circle. Based on a threshold, it is determined whether the edge end point is close enough to the vertex to form a connection. Since an endpoint can be connected to at most one vertex, we iteratively update the connection and finally end up pairing the edge with the vertex that has the minimum distance to the circumference of the vertex (or circle). Constructing an algorithm for arrow-edge connections was much more difficult. Since an arrow is represented with a triangle, choosing the correct orientation of the triangle is very important. Finding the intersection point of the edge and a side of the triangle helps in getting the correct orientation. Adding an extra point after the original endpoint made sense as it brought the edge closer to the correct corner of the arrow. This made it easier to do a threshold-based check on the distance between the new endpoint and the correct corner or the new endpoint and the midpoint of the intersecting side to determine the validity of the connection. Here also we have the constraint of an arrow being able to connect to at most one edge endpoint. The new endpoint was added utilizing the nearby slope of the edge as follows: * Equation of new point using slope*

\subsection{Cost Calculation}

We followed algorithm 1 in (cite) to calculate the cost of a hand-drawn graph's interpretation using our algorithm. The cost is nothing but the number of subgraphs the algorithm gets wrong. This means that the lower the cost, the better the performance. When we say subgraph, we mean the subgraph obtained after adding the strokes inputted in step $i$ to the subgraph that was locked-in at step $i-1$. The concept of a locked-in graph is important in an online setting as this means that the subgraph that has already been interpreted cannot be changed. This follows from the online goal of analyzing the graph at most five components at a time. A fundamental struggle in this step was to come up with a good function to check whether the interpreted subgraph is isomorphic to the intended subgraph as required by the algorithm. Since it is not tractable to try all permutations of vertices, we opted to focus on a few well-known heuristics. Checking for the number of vertices and edges and the vertex degree sequences of both the graphs often works well for simple graphs. We used just these three conditions to detect isomorphisms. Other heuristics like checking whether the degrees of adjacent vertices match were complex and hence we did not try them. We also checked isomorphisms only for the undirected versions of directed graphs as it was very unlikely in our case that directed graphs are isomorphic, but their undirected versions are not.\\

Training is done by post-processing the sketches of users. Since we are implementing the online method now, our recognition is triggered a fixed time after the user stops sketching. We do not want the already graph to change. Hence, the "locked-in" portion of the graph doesn't change here. New connections are made between the newly drawn components and the locked in components by a recognizer. The recognizer maintains a sequence of added strokes and the corresponding sub-graphs. For every pair, we select the most promising candidate graph according to our scores. However, if an isomorph exists, we select the isomorph as the representation of the intermediate subgraph. If there exists no isomorphism, then we penalize all the remaining pairs of new strokes and subgraphs.\\

After successfully completing the online method, we will work on the offline method. There are some fundamentally different aspects to the offline method like the recognizer will only act on the graph after completion and connections do not have to maintained to a locked-in graph. We aim to have an effective offline interpretation as our ambitious goal.\\

\subsection{Experiments and Results}

\begin{table}[!htb] %% use b,t,h
    \centering
    % you can use l,r,c for left-aligned, right-aligned or centered columns
    \begin{tabular}{lrr}
    \toprule
       Graph & Avg. Cost & Steps \\ % use ampersands to separate columns and \\ for end of rows
         \midrule
3  & 2.4 & 3  \\ 
4 & 2.0 & 3 \\
6 & 1.6 & 2 \\
7 & 2.8 & 3 \\
9 & 2.6 & 3 \\
11 & 3.6 & 4 \\
13 & 5.6 & 6 \\
         \bottomrule
    \end{tabular}
    \caption{Average cost of interpreting 7 graphs with 5 trials each}
    \label{tab:table_cost}
\end{table}

We conducted our experiments using the end-to-end pipeline (shown in the IPython notebooks in the repository) of segmentation, classification, and interpretation for each step, where a step signifies a set of user strokes that are analyzed by this pipeline and added to the locked-in graph to form a new locked-in graph. We selected graphs 2,3,4,6, 7, 11 and 13 from the mechanical turk experiment 1 as listed in the (cite). Each graph was drawn five times with five components added in each step, except the last step where five or less components were added. The five components being added in a particular step of a graph were not the same across trials, i.e., we didn't always add the same five components for step 1 of graph 7. Table shows the average cost of interpreting each graph. Naturally, the average cost is going to be more for graphs that require more steps to be drawn. Hence, the average total cost does not particularly help in comparing the algorithm's performance across graphs. Hence, we constructed a plot in figure which shows the average cost of steps 1 and 2 for all the graphs. Since not incurring cost at step $i$ requires the cost at step $i-1$ to be 0, the average cost for step 2 will always be more than or equal to the average cost of step 1 for a particular graph. We also trained within the parameter space stated in (cite) to obtain the best coefficients of the terms used in the candidate subgraph scoring function. Several combinations yielded the lowest cost including A = -1.1, B = 0.3 and C = -0.4.

\begin{figure}[!htb]
    \centering
    \includegraphics[width=1.0\textwidth]{./img/cost_plot.pdf}
    \caption{average cost of step 1 and step 2 for the 7 graphs}
    \label{fig:cost_plot}
\end{figure}

\subsection{Analysis}

The table shows that our pipeline did not perform exceptionally well. This was mainly because the cost, as defined in \cite{daly2015hand}, is a very harsh penalty. It is binary with no partial reward for some or most of the subgraph being correct. Since the correctness of an interpreted subgraph is determined by whether it is isomorphic to the intended subgraph, the cost at a particular step will 0 if and only if there is an isomorphism between the two. Other issues that affect our algorithm is the fact that we reuse the thresholds from (cite). These thresholds may not be appropriate for our drawing setting in iPyCanvas \cite{ipycanvas}. Training for finding our own values for the coefficients in the scoring step was not very helpful. A reason could be the search space for the parameter grid search was not right. Another potential weakness is probably the inconclusive isomorphism function. Figure shows that subgraphs with vertex and edges are more easily interpreted than those with loops and arrows. This was expected as circles for vertices and smooth lines for edges are much easier to break into segments and classify than self-loops and triangles for arrows.
	
	\section{Conclusion and Future Work}
	In this project, we worked on the online graph recognition problem. For offline recognition, we tried to use the same pipeline by converting an offline graph to a sequence of coordinates. But this approach was not successful as the current pipeline heavily relies on the ordering of the coordinates. The offline recognition problem can be modeled as a learning problem and deep learning approaches can be used to solve it.\\
	
	The segmentation algorithms discussed in this report heavily rely of hyperparameter tuning. The values of these parameters can vary from person-to-person. Rather than finding the parameter values that work for everyone, we can tune these parameter for an individual by taking feedback from the user. The speed-based segmentation can be explored further. To reduce the false-positives, an abnormality computation step similar to the curvature-based segmentation can be added to overcome the noise.\\
	
	For classification, we found that it was a task well-suited to multiple pipeline configurations, only needing to adjust the input format. The geometric feature approach described in the original paper was the most effective. The problem is simple enough that attempting convolutional methods introduces significant complexity for minimal increase in efficacy.\\
    
  Classification is simple enough not to need an approach more complex than the features discussed here, but it is closely related to segmentation. Future work could include an object detection method to perform both tasks simultaneously.\\

  Interpretation depends heavily on segmentation and classification for accuracy. The main portions of interpretation are establishing vertex-edge and arrow-edge connections, scoring candidate graphs, and calculating the cost of an interpreted graph. From our results, we can see that vertices and edges were easier to deal with than arrows and loops. Future work would involve improving on the isomorphism function that we have. Another potential area is to experimentally determine threshold values. Another obvious area of development is building the offline method. There are some fundamentally different aspects to the offline method like the recognizer will only act on the graph after completion of all strokes and intermediate subgraphs do not have to be maintained in a locked-in graph. We found out that the cost function is not the best metric to evaluate our methods. It assigns full cost even to a graph with just one incorrect component. Thus, developing another quantitative performance evaluation metric for hand-drawn graphs is something that we want to work on in the future.\\
    
  In the current pipeline, the output needs to be evaluated by inspecting probabilities and costs.  It can be replaced with a user interface to visualize the output. It is a required feature for the end-user and will significantly reduce the time to analyze the performance of this pipeline.
	
	\section{Member Roles}
	Neeraj: segmentation, Bryan: classification, Debopam: interpretation and training. All of us worked on integrating the subparts into a whole. We used Github to maintain and share data (\href{https://github.com/neerajgangwar/graph-recognition}{Github Repository}). We met once a week to discuss updates and issues.
	
	\printbibliography[title=References]
	
\end{document}
